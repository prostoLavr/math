\documentclass[a4paper,12pt]{article}
\usepackage{cmap} % Поиск в документе
\usepackage[T2A]{fontenc}  % Внутреннюю кодировку TeX
\usepackage[utf8]{inputenc} % Кодировка файла
\usepackage[english,russian]{babel}  % Языки
\usepackage{amsmath,amsfonts,amssymb,amsthm,mathtools} % Математические плюшки
\usepackage{graphicx} % Изображения
\usepackage{pgfplots} % Графики
\pgfplotsset{width=10cm,compat=1.9}
\pgfplotsset{samples=100}


\author{Лаврентий Наумов}
\title{Формулы приведения}
\date{\today}

\theoremstyle{definition}
\newtheorem{definition}{Определение} % Создание теорем
\newtheorem{principle}{Правило} % Создание теорем
\newtheorem{problem}{Задача}
\theoremstyle{theorem}
\newtheorem{theorem}{Теорема}


\begin{document}
    \maketitle
    \section{Приведение функций sin и cos}
    \begin{tikzpicture}
        \begin{axis}[
            domain=-6:6,
            xlabel = {$x$},
            ylabel = {$y$},
            minor tick num = 1,
            width=400px,
            legend pos=outer north east
            ]
            \addplot[blue] {sin(deg(x))};
            \addplot[red] {cos(deg(x))};
            \addplot[green] {-sin(deg(x))};
            \addplot[orange] {-cos(deg(x))};
            \legend{$\sin(x)$,$\cos(x)$,$-\sin(x)$,$-\cos(x)$}
        \end{axis}
    \end{tikzpicture}
	Заметим, что интервал между соседними точкмми экстремума равен $\frac{\pi}{2}$. Следовательно, перейти из одной функции в сосденюю можно при помощи прибавления к аргументу функции и вычитания из него  $\frac{\pi}{2}$. Вспомним, что движение в правую часть - это вычитание из аргумента функции, и движение в левую часть - сложение с аргументом функции. Следовательно мы можем разместить пики фукнций в некотором порядке $\sin(x)$, $\cos(x)$, $-\sin(x)$, $-\cos(x)$ и перемещается по функциям с прибавлением $\frac{\pi}{2}$ в левую сторону и с вычетанием $\frac{\pi}{2}$  в правую. Например, $\cos(x -\frac{\pi}{2})=\sin x$. Цикл функций замкнутый, поэтому переход $\cos(x-\frac{3\pi}{2}) = -\sin x$ не является ошибочным. В данном примере, мы вычитали $\frac{3\pi}{2}$, но ничего страшного нет, ведь это просто вычитание $\frac{\pi}{2}$ три раза. Аналогично, переход $-\sin(x + \pi) = - \sin(x + \frac{2\pi}{2}) = \sin x$ корректен.



\end{document}
